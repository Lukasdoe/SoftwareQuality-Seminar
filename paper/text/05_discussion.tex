\section{Discussion}
\subsubsection{Maintenance}
Many of the problems and sources of unsafety introduced by the examined \ac{rts}-tools are caused by
incompatibilities of the tools with modern language features. Especially \emph{Ekstazi}, being one of the
most promising examined tools, is outdated and cannot be used in projects in its current state.
This abandonment of \ac{rts}-tools, such as \emph{HyRTS}, \emph{Ekstazi} and \emph{GIBstazi}\footnote{\emph{STARTS} has recently
    been maintained and is therefore not considered abandoned.}, leads to a market full of
unusable \ac{rts} solutions. Research should focus more on updating and improving existing open
source solutions than developing new projects from scratch that are not going to be maintained.

\subsubsection{Usability of the Examined Tools}
Looking at the discovered sources of unsafety and their occurrence in current Java software development,
it is clear that none of the examined tools should be used without in-depth knowledge of their
downsides. Because of the continued maintenance on the \emph{STARTS} program, we see high future potential in
this static \ac{rts}-tool. However, the current latest official release (version 1.3) is not functional. The
previously described state of dynamic \ac{rts} for Java is alarming. The examined tools are all
poorly maintained. The most up-to-date tool is \emph{OpenClover} with its last release in October 2019.
The dynamic tools also act surprisingly unsafe, compared to the possibilities they have for monitoring changes.

\subsection{Limitations}
This paper does not claim to include all sources of unsafety that occur when using one of the
examined tools. They may contain bugs that we did not find. Other external tools or libraries, for example other
\ac{di} frameworks than Spring or Guice, could also cause undiscovered unsafety. We concentrated on
evaluating the major causes of unsafety that are likely to occur in real world software projects.

The results of the repository scanning are limited by the number of scanner modules, our methodology
and the selection of projects. We did not scan for all discovered sources of unsafety. Our scanning
approach trades simplicity for accuracy, meaning the results are prone to including false positives
and false negatives.
Every line of text in a Java source code file was treated as runnable code, only plaintext files
were analyzed. Many well known Java projects were excluded from this study, because we use the GitHub stars
feature to determine the popularity of projects and they did not have enough stars to be included in
the top 100 repositories.

We excluded all projects that do not use the Maven build system. This was necessary to enforce a
uniform project structure for the external files scanner module. The Maven tool is used by
approximately 60\% of all Java projects~\cite{project_management_tools_report}, making it a viable prerequisite for scanned projects. 
