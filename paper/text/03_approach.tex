\section{Methodology}
% High level overview over the methodology, design
% Step 1: identify sources of unsafety with poc-repos

% Step 2: scan wild open source projects for these unsafeties
% Which data did I test it on? -> Study Objects
% \subsection{Study Objects}

% \begin{itemize}
% \item STARTS (Version 1.4-SNAPSHOT\footnote{This version was built from commit
% e1d29be2958ec27fac12e6c8611577fce5a73e40 from the tool's GitHub repository~\cite{starts_tool},
% because the newest version (edu.illinois:starts-maven-plugin:1.3) in the Maven Central Repository is
% not functional.})
% \item Ekstazi (Version 5.3.0)
% \item HyRTS (Version 1.0.1)
% \item GIBstazi (Version 3.5.7)
% \item OpenClover (Version 4.4.1)
% \end{itemize}

The examined study objects shown in Table~\ref{table:rts-tools} each represent a different approach
to \ac{rts}. \emph{Ekstazi} registers test case dependencies on a file-level while \emph{GIBstazi} and HyRTS
use hybrid approaches, on a module-/file-level and file-/method-level granularity. \emph{OpenClover}
performs intrusive, static code instrumentation by altering the program's source code~\cite{clover_documentation,java_instrumentation}. The
other dynamic \ac{rts}-tools perform dynamic bytecode instrumentation with an agent that intercepts
and changes bytecode during the class loading process~\cite{hyrts_paper,ekstazimain,java_instrumentation}.

\begin{savenotes}
    \begin{table}[H]
        \caption{\textbf{Study Objects:} Examined \ac{rts}-tools with their corresponding versions.}\label{table:rts-tools}
        \centering
        \begin{tabular}{l | c}
            \hline
            Name                                     & Version                                                  \\
            \hline
            \emph{STARTS}~\cite{starts_tool}         & 1.4-SNAPSHOT\footnote{This version was built from commit
                e1d29be2958ec27fac12e6c8611577fce5a73e40 from the tool's GitHub repository~\cite{starts_tool},
                because the newest version (edu.illinois:starts-maven-plugin:1.3) in the Maven Central Repository is
            not functional.}                                                                                    \\
            \emph{Ekstazi}~\cite{ekstazi_tool}       & 5.3.0                                                    \\
            \emph{HyRTS}~\cite{hyrts_tool}           & 1.0.1                                                    \\
            \emph{GIBstazi}~\cite{gibstazi_tool}     & 3.5.7                                                    \\
            \emph{OpenClover}~\cite{openclover_tool} & 4.4.1                                                    \\
            \hline
        \end{tabular}
    \end{table}
\end{savenotes}

% What makes an rts tool a good rts tool -> Evaluation criteria

% Why am I doing this? -> Research Questions
\subsection{\acp{rq}}
The \acp{rq} for this seminar paper are the following:
\begin{itemize}
    \item \textbf{RQ1} What sources of unsafety exist for the examined \ac{rts}-tools?

    \item \textbf{RQ2} What are the differences in safety of the examined tools, in the context of the previously
          identified sources of unsafety?

    \item \textbf{RQ3} How can potential for unsafety be automatically identified in code changes
          without dynamic program analysis?

    \item \textbf{RQ4} In which quantities do the identified sources of unsafety occur in real world software projects?
\end{itemize}

\subsection{Identification and Documentation of Sources of Unsafety}\label{ssec:pocs}

\begin{definition}\label{def:unsafety}
    A source of unsafety consists of a combination of an existing code base $S_0$ and changes
    to the source code $C$. The application of the changes $C$ to $S_0$ leads to the new program revision
    $S_1$. Every valid source of unsafety makes at least one of the examined
    \ac{rts}-tools behave in an unsafe manner. This means that either test cases whose behavior is
    affected by the changes $C$ are excluded or the tool interferes with the testing process in another
    way that leads to unwanted deviations in the test results.\footnote{The definition of a source of
        unsafety is based on the definition of safe \ac{rts} introduced by Rothermel and Harrold (1997)~\cite{safe_definition}.}
\end{definition}

We built model Java projects for all scenarios that could be sources of unsafety.\footnote{Our
    supplementary material containing all model projects and experiment results can be found in~\cite{poc_github}.} We refer to
these model projects as \ac{poc} repositories. They were combined as modules in a Java project,
managed by the Apache Maven software management tool. This enabled us to define a set of Maven
profiles that activate the \ac{rts}-tools on a top level, reducing the configuration overhead for each
submodule.

All \ac{poc} repositories are executed and built using the Java Standard Edition (SE) Development
Kit (JDK) version 8u291\footnote{The Java SE is used by about 70\% of all Java projects, with 79\%
    of them using Version 8.}. The examined tools either support this version or do not specify a
supported JDK version. Besides Maven (version 3.8.3), we used the JUnit unit testing framework in
version 4.12 for the dependency injection tests and in version 4.10 for all other tests. \emph{Ekstazi} and
\emph{OpenClover} were combined with the Maven Surefire Plugin (version 3.0.0-M5) as
instructed by their documentations.

The following subsections each describe potential sources of unsafety. The supplied implementation
details describe the main structure of the \ac{poc} repositories that were used to test
the corresponding hypotheses.

\subsection{Dynamic Dispatch as a Source of Unsafety}

The Java language's polymorphic type system allows the Java runtime to select a more specific
implementation of a called method, if this implementation is available. This behavior was already
identified to cause unsafety of the \emph{OpenClover} tool, but was not evaluated on all examined \ac{rts}-tools~\cite{unsafety_eval}.

\begin{hypothesis}\label{hyp:dyn_dis}
    Changes that lead to a different dynamic dispatch are not recognized as dependency
    changes. This means that changes to an object's implementation are undetected, as long as the static type of
    the variable storing the object is of a less specific superclass.
\end{hypothesis}

\subsubsection{Implementation \arabic{hypothesis}}
We created a \ac{poc} repository containing two classes. Class $A$ that implements a method $m$ and
class $B$ that inherits from $A$ and does not implement the method $m$. A test case $t$ initializes
a variable $b$ of type $A$ with an instance of the class $B$. The test compares the return value of $b.m$
with the expected return value of $A.m$. They are equal because $B$ does not implement $m$ and thus
no dynamic dispatch was performed. The
changeset $C$ adds the definition of a method $m$ to class $B$, overwriting the inherited version
from class $A$. This new method returns a different return value than $A.m$.

\subsection{External Files as Sources of Unsafety}\label{ssec:external_files}

The examined \ac{rts}-tools concentrate on identifying code dependencies and perform test exclusion
mostly based on source code changes. This however does not work with programs that have side effects and
dependencies outside of their available source code. The examined tools \emph{GIBstazi} and \emph{Ekstazi} are,
according to the their research papers, especially built for identifying such external file
dependency changes~\cite{ekstazimain,gibstazi_paper}.

\begin{hypothesis}\label{hyp:external_files}
    Changing an external non-Java file is not correctly
    detected as a change of test case dependencies.
\end{hypothesis}

\subsubsection{Implementation \arabic{hypothesis}}
We chose to test external files that are in the standard directory
for program code dependency files. This directory is specified by the Maven management tool to be
\texttt{src/main/resources}, while the source code should be in \texttt{src/main/java}. A test case $t$, directly
or indirectly, reads the contents of a plain text file at
\texttt{src/main/resources/test.txt} using an \texttt{InputStream} from the Java standard io library
(\texttt{java.io}). The changeset $C$ simply contains changes to the contents of the dependency text file.

\subsection{Configuration Files as Sources of Unsafety}\label{ssec:config_files}
Java projects typically use \texttt{.xml} or \texttt{.property} files to configure the behavior of libraries and additional
tools. Famous examples are the \texttt{pom.xml}, \texttt{gradle.properties} or \texttt{build.xml} files that
configure the most popular open source Java dependency management tools Maven, Gradle and Ant~\cite{project_management_tools_report}. We differentiate
between external files and configuration files. The former are explicitly accessed through program
code. The latter are used implicitly by external tools or libraries.

\begin{hypothesis}\label{hyp:config_files}
    Changes in library versions and other changes introduced through external configuration files are
    not correctly recognized by the examined \ac{rts}-tools.
\end{hypothesis}

\subsubsection{Implementation \arabic{hypothesis}}
We simulated the update of a library dependency to a new minor version. For this example, we updated
the jackson JSON parsing library~\cite{jackson_json_lib} from version 2.11.4 to version 2.13.0. $S_0$ consists of
a test case $t$ that converts an instance of a \texttt{Thread} class' class loader to a string and asserts
the outcome. To transition to $S_1$, we only updated the library version and
did not change the source code. The new library version had introduced changes in the object to string
mapping and should therefore cause the test $t$ to fail. Though, the test case would not be executed by the
\ac{rts}-tools, if~\ref{hyp:config_files} is true.

\subsection{Reflections as Sources of Unsafety}

Higher order class access through Java's meta class \texttt{Class} is often cited as a source of unsafety
in regression test selection~\cite{ekstazimain,starts_paper,unsafety_eval,prestarts}. The STARTS
\ac{rts}-tool's authors identify this reflective access as the only source of unsafety, when
comparing their tool to \emph{Ekstazi}~\cite{starts_paper}. The \emph{HyRTS} tool is built, according to its
authors, to overcome this weakness of static \ac{rts} by using a hybrid, dynamic
approach~\cite{hyrts_paper}. Reflections are a powerful and commonly used technique. The factory pattern
is an example of a common software engineering use case that often uses reflections.

\begin{hypothesis}\label{hyp:reflections}
    The examined \ac{rts}-tools do not recognize reflective instantiation of a class. They do not create
    dependencies to objects that were created from their \texttt{Class} meta object, without explicitly importing
    the class.
\end{hypothesis}

\subsubsection{Implementation \arabic{hypothesis}}
We declared a class $A$ that is not imported to the test case $t$'s class file. Class $A$ declares a
method $m$ which will later be tested. The test case $t$ acquires $A$'s meta class instance via
\texttt{Class.forName("A")} and stores it in a variable $a$. Tests are executed on a new instance of
$A$, created either via \texttt{a.newInstance()} (deprecated since Java
9~\cite{java_9_class_api_docs}) or via $A$'s constructor
(\texttt{a.getDeclaredConstructor().newInstance()}). In $S_1$ we changed the implementation of the
class $A$ to fail test case $t$. The test should not be executed by the
\ac{rts}-tool, though, if our hypothesis holds. It is important how we access the meta
class object. Although we could also call \texttt{A.class}, this could create a solid
dependency on class $A$, hiding any potential for unsafety.

To display a common use case of these reflective accesses, we also implemented a simple factory
pattern using reflections.

\subsection{Static Initializers as Sources of Unsafety}

Java classes support the declaration of static initialization blocks. These are code segments that are
executed once on class initialization. This process is either triggered by direct class access or
by certain reflective access methods~\cite[12.4.1]{java_8_spec}.

\begin{hypothesis}\label{hyp:static_init}
    \ac{rts}-tools do not monitor side effects that are caused by static
    initializers on class initialization. Classes initialized without a direct call to a constructor or static method can alter the
    program's behavior and test results without being detected.
\end{hypothesis}

\subsubsection{Implementation \arabic{hypothesis}}
Class $A$ contains an uninitialized static public boolean field $A.b$. Class $B$ declares a static initialization
block that contains code that sets $A.b$ to equal $true$. The test case $t$ triggers the initialization of $B$ using the
reflective access \texttt{Class.forName("B")}. $t$ asserts that $A.b$ equals $true$. This is the
base state $S_0$. For $S_1$, we changed the static initialization block in
$B$ to set $A.b$ to equal $false$. If~\ref{hyp:static_init} is true, this change of the dependencies of test case $t$ is not
recognized by the \ac{rts}-tool and $t$ is not executed.

\subsection{Dependency Injection as a Source of Unsafety}

Many professional Java projects utilize the technique of \ac{di}. To reduce coupling
between components of a program, an injector inserts initialized instances of required dependencies
into methods, constructors and fields. This programming pattern is one possible realization of
``inversion of control''\cite{inversion_of_control_coiner}.~\cite{inversion_of_control_di}

\ac{di} in Java is commonly used through third party open source libraries such as Spring, Google's Guice or
CDI~\cite{java_cdi_jsr299} implementations. For the following study, we concentrate on the former two libraries, being the ones used in
some of the repositories from the open source project selection, which is later introduced in
Section~\ref{sec:foss_search}.

% We expect \ac{di} to be a significant cause of unsafety because of the amount of control and logic
% that is shifted from explicit declarations in the program's source to implicit behavior of external
% libraries. This makes it one of the most important sources of unsafety, as dependency injection is
% common among big projects with complicated dependency structures.

Because of the strong static typing system of the Java programming language,
parameters and attributes have to have a strict static type. Dependency injection libraries use Java
interfaces to provide different implementation with equal signatures.

Dependencies are injected based on a mapping of implementations of interfaces to dependency
identifiers. This mapping is either explicitly defined in program code or in external \texttt{.xml}
configuration files. The Spring \ac{di} framework also offers automatic component scanning,
automatically searching for dependency implementations in specified class paths. Methods and fields
that require injection are marked using Java annotations.

\begin{hypothesis}\label{hyp:dep_inj:source}
    Assume that the mapping of dependencies was explicitly defined in program code.
    Changes to the source code of injected dependencies can then not be tracked reliably by the
    examined \ac{rts}-tools.
\end{hypothesis}

\subsubsection{Implementation \arabic{hypothesis}}
The implementation for the Spring framework is the same as described in
Section~\ref{sssec:impl:class_path_scanning} with the difference that we do not introduce a new
implementation for $D$ in class $C$, but we change the source code of the implementation defined in
class $B$.

For Guice dependency injection, no main application context is required, as dependencies are
injected using an instance $i$ of the \texttt{Injector} class. This means that it is sufficient to annotate
the dependency interface $D$ with the \texttt{@ImplementedBy(B.class)} annotation, referring to
an implementation $B$ of the interface $D$. Because
dependency implementations need to be injected into objects, we define a helper class $H$ that
contains a Field of type $D$, annotated with the \texttt{@Inject} annotation. The test case $t$
asserts the behavior of an instance of $H$, initialized using the \texttt{i.getInstance(H.class)}
method. For $S_1$, we simply modify the source code of class $B$.

\begin{hypothesis}\label{hyp:dep_inj:external}
    Assume that the mapping of dependencies was declared
    in an external configuration file.
    Then, changes to the source code and changes of the dependency mapping are not
    detected by the examined \ac{rts}-tools.
\end{hypothesis}

\subsubsection{Implementation  \arabic{hypothesis}}
The hypothesis depends on the inability of the examined \ac{rts}-tools to detect changes of external
files. A corresponding \ac{poc} repository was already implemented and described in
Section~\ref{ssec:config_files} for hypothesis~\ref{hyp:config_files}.

\begin{hypothesis}\label{hyp:dep_inj:collections}
    The \ac{rts}-tools do not track changes of injected collections of dependencies.
\end{hypothesis}

Both \ac{di} libraries support automatic aggregation of multiple implementations of the same interface into a collection
of implementations.

\subsubsection{Implemenetation  \arabic{hypothesis}}
Collection injection with the Spring framework can be achieved by choosing one of the approaches
shown in Section~\ref{sssec:impl:class_path_scanning}. The type of the tested field $f$ is
changed from \texttt{D} to \texttt{Collection<D>}. The Spring framework now injects a collection
of all available implementations for the dependency $D$ into the field $f$. To transition to
$S_1$, we perform the same actions as described in Section~\ref{sssec:impl:class_path_scanning}, but without annotating the new implementation with the
\texttt{@Primary} annotation. The added implementation is automatically added to the injected
collection, causing test case $t$ to fail. If the hypothesis holds, this dependency change is not
detected by the \ac{rts} tools.

The same principles apply to the implementation for the Guice library. Dependencies have to be
explicitly bound to implementations, though. The Guice injector is configured from a subclass $A$ of
type \texttt{AbstractModule}. A possibility to add objects for the collection injection
is to create multiple provider methods, returning different implementations of the injected
interface $D$. These methods have to be annotated with the \texttt{@ProvidesIntoSet} annotation. We only
declare one such provider method for the base state $S_0$. To transition to
$S_1$, we just add another provider method, adding a new implementation to the injected
set.

\begin{hypothesis}\label{hyp:dep_inj:scan}
    Dependencies collected using class path scanning without an explicit mapping are not tracked as test
    dependencies by the examined \ac{rts}-tools.
\end{hypothesis}

\subsubsection{Implementation \arabic{hypothesis}}\label{sssec:impl:class_path_scanning}
Automated class path scanning is only available in the Spring framework~\cite[1.10]{spring_manual}\cite{guice_jit_bindings}.
% The Guice library requires
% additional extensions like Netflix Governator for automated dependency collection. However,
% Governator deprecated its automatic dependency implementation scan feature, because of the risk of
% unintended dependency instantiation~\cite{governator_wiki_auto_bind}.

Class path scanning with Spring can be achieved in two ways. For the first \ac{poc} repository, we
use implicit configuration scanning. Class $A$ in package $P$ is the main application class, the entrypoint of a
Spring application. It is annotated with the \texttt{@EnableAutoConfiguration} annotation, which is
also included in the commonly used \texttt{@SpringBootApplication} annotation. Package $P$ contains a configuration
package $p$ with a configuration class $B$ that is annotated with the \texttt{@Configuration}
annotation. The test class $T$ is annotated with \texttt{@RunWith(SpringRunner.class)} and
\texttt{@SpringBootTest(classes = A.class)} annotations to enable dependency injection. Test case
$t$ asserts the behavior of a field $f$. This field is annotated with the \texttt{@Autowired}
annotation. Spring automatically injects the dependency $D$ whose implementation is defined in class $B$. To transition
to $S_1$, we declare another configuration class $C$ in package $p$. This class defines an alternative
implementation of $D$ and markes it with the \texttt{@Primary} annotation. This way, it is
prioritized over other alternatives. This new implementation $C$ of $D$ should make $t$ fail on
execution.

Another option for class path scanning with Spring is to replace the \texttt{@EnableAutoConfiguration}
annotation of class $A$ with a \texttt{@Configuration} and \texttt{@ComponentScan("P.p")}
annotation. This avoids the overhead produced by a full Spring Boot application. The test class $T$ does not
need any annotations any longer, because we use the \texttt{AnnotationConfigApplicationContext(A.class).getBean(d.class)}
function in $t$ to retrieve implementations for dependencies. The transition to $S_1$
stays the same, with the same expected test outcome.

\subsection{Runtime Instrumentation as a Source of Unsafety}\label{ssec:runtime_instrumentation}

The examined dynamic \ac{rts}-tools need to collect their dependency information during execution.
They use different instrumentation mechanisms to capture call trees and resource accesses. \emph{Ekstazi} and
\emph{HyRTS} both perform dynamic runtime instrumentation by augmenting java bytecode during class
loading~\cite{ekstazimain,hyrts_paper,java_instrumentation}. \emph{OpenClover} on the other hand performs static runtime
instrumentation by inserting additional instrumentation code into the java source files~\cite{clover_documentation,java_instrumentation}.

\begin{hypothesis}\label{hyp:instrumentation}
    Runtime instrumentation leads to differences in test results when executing the code with or without
    the dynamic \ac{rts}-tools. Although this does not concern dependency collection, every undesired
    deviation of test results is treated as unsafe behavior, according to the definition~\ref{def:unsafety}.
\end{hypothesis}

\subsubsection{Implementation \arabic{hypothesis}}
To showcase the problems with runtime instrumentation, we combined already discovered problems of
\emph{OpenClover}~\cite{unsafety_eval} with additional problems caused by its alterations to the compiled
class files. We also found incompatibilities of the used \emph{Ekstazi} version with the new methods
introduced in the \texttt{Class} meta class with Java 8. The problems are caused by a monitoring class
that \emph{Ekstazi} injects during class loading for method calls on the \texttt{Class} meta class. We call the
set of method calls that are affected $M_{CL}$
(documented in table~\ref{table:runtime_instr_methods}). For testing the hypothesis on the identified
methods, we create a test case for every method $m \in M_{CL}$.

\subsection{Evaluation of Proof of Concept Repositories}

The process of \ac{rts} depends on a set of changes of the program code. Therefore, we run every model
project's tests
twice. Once with the base state $S_0$ and once with the changed version $S_1$ to
simulate creating a new revision. We started by running each \ac{poc} repository's tests before
($S_0$) and
after ($S_1$) the simulated changes were applied
without any activated \ac{rts}-tools.
This dry run yields two test reports that were later used to identify
deviations from the expected outcome when testing with activated \ac{rts}-tools.

Using the earlier mentioned Maven profiles, we executed both revisions' tests ($S_0$ and
$S_1$) of every \ac{poc} repository with every \ac{rts}-tool. The test runs supplied all
data required to filter out the valid sources of unsafe behavior. A repository implements such a valid source if at
least one of the examined tools caused a test result that differs from the dry run
result.\footnote{Apart from the difference caused by the test cases that were intended to be
    excluded by the tools.}

With the filtered \ac{poc} repositories, we had a solid overview of the most important cases
that induce unsafe behavior of the examined \ac{rts}-tools. This however does not yet show the
impact of this unsafety in practical software engineering projects as we could have identified purely
hypothetical sources of unsafety that do not occur in everyday programs.

\subsection{Search for Unsafety in Open Source Projects}\label{sec:foss_search}

\subsubsection{Project Selection}

In order to estimate the impact of the discovered sources of unsafety on real world software
engineering, we performed a broad automated search over the most popular open source Java projects
on GitHub. The base set of projects was acquired with the Python
library PyGithub which uses the GitHub REST API as a data backend~\cite{pygithub}. The result was then further
filtered.
% to only contain software projects that are candidates for utilizing one of the examined
% \ac{rts}-tools in the future.


To be included in the final set of the 100 examined projects, a repository has to meet the following
criteria:

\begin{itemize}
    \item Over 50\% of the repository's content has to be Java program code.
    \item The last code changing activity was after or on the 06/01/2020.
    \item The repository is not forked from another repository.
    \item The repository has at least 100 stars\footnote{Stars are a measure of popularity on the GitHub
              platform. Users can attribute stars to repositories to show their appreciation or to store
              the project in their list of starred projects for later reference.}.
    \item The repository contains a branch called ``main'' or ``master''\footnote{GitHub, together with
              many other software projects~\cite{ZDNet_master_main}, decided in 2020 to revise offensive or inappropriate technical terms.
              The hosting provider changed its default git repository branch name from ``master'' to ``main''~\cite{github_main_master}.}, having a history of least 101
          commits. This branch is considered to be the default branch.
    \item The repository is not archived.
    \item The repository is publicly accessible and readable.
    \item The description of the repository does not contain the words ``tutorial'', ``example'' or ``sample''.
    \item The repository contains a Maven configuration file (``pom.xml'') in its root folder.
\end{itemize}

In addition to these criteria, we also excluded repositories that contained too few JUnit test
cases, as these repositories do not represent serious software development projects. This criterion
for project selection is common in research and was also used for evaluating the Ekstazi
\ac{rts}-tool~\cite{ekstazimain}. We therefore
examined the state of the repository at 100 commits before the current default branch's latest commit and counted all
appearances of the keywords \texttt{@Test} for JUnit 4 and 5 test cases and \texttt{extends TestCase} for
JUnit 3 test cases. We are aware that we might underestimate the amount of JUnit 3 test cases with
this method. However, other means of counting would make this filter unnecessarily complicated and 100 test cases
are already an arbitrarily chosen, lower limit which should be easily exceeded.

The remaining set of software projects was sorted by the amount of stars, which users attributed to the
projects. We included the top 100 projects of the resulting list in the further study.

\subsubsection{Commit Selection}

We searched for the previously identified patterns for sources of unsafety in a filtered subset of
the latest commits in the selected projects. Only the latest 100 commits on the default branch of
the repository (which
introduce new changes) have been selected.
% Additionally, the commit has to have a parent commit.
% This is required to have a base state $S_0$ for the changes introduced in the commit.
% Without this base repository state, any applied \ac{rts}-tool would have to execute every regression
% test case anyways. However, this restriction is already enforced when selecting the projects and
% therefore makes every additional check obsolete.

\subsubsection{Automated Repository Scanning}\label{sssec:repo_scanning}

Not all identified sources of unsafety are detectable using a simple and efficient scanning
technique, without performing a dynamic analysis of the projects. We opted not to perform
such a dynamic, in depth analysis. Most of the previously collected scanning candidates are not executable
without individual configuration. Sorting out projects that do not
work with dynamic analysis tools artificially filters the pool of examined projects,
distorting the study's results.

We also did not scan for changes of configuration files, as these must occur at some point in every selected
repository, making an in-depth analysis unnecessary.

For the efficient scan of the open source projects, we built a
modular Python application. Scanner modules are classes that receive a commit and decide whether
this commit should be treated as a source of unsafety. The main application iterates over the set of
projects and their commits. Using the PyDriller library~\cite{pydriller}, we clone every project's
repository. To recreate the project state after each commit, we use the PyDriller git integration to
programmatically check out each commit. Then, we call each scanner module to test for its source of
unsafety in the program files. This process is repeated for the latest 100 commits of the previously
selected 100 projects. The results are collected as \texttt{Dataframes} and stored in \texttt{.csv}-files via
the pandas data manipulation library~\cite{pandas}.

We created the following scanner modules to automatically detect sources of unsafety:

\begin{itemize}
    \setlength\itemsep{1em}
    \item \textbf{External Files:}\\This scanner reports a source of unsafety if a file was
          changed in the commit and its file path lies in either a resources or filter directory. These
          directories' paths were chosen according to
          the directory layout recommended by the Maven software management tool~\cite{maven_directory_layout}.

    \item \textbf{Dependency Injection:}\\The dependency injection scanner holds a set of regular
          expressions.
          They identify keywords that are required for unsafe behavior with either the Spring or the Guice
          framework (Table~\ref{table:dep_inj_keywords}). We search for these expressions in the changed lines
          of Java source code of each commit. Every match is a potential source of unsafety, as it may not be
          recognized by one of our examined \ac{rts}-tools.

    \item \textbf{Runtime Instrumentation:}\\We perform a search for problematic methods which were found in the evaluation of
          Section~\ref{ssec:runtime_instrumentation} (Table~\ref{table:runtime_instr_methods}).
          With the program state right after the commit checked
          out, we read every Java code file in the project's directory and test if it contains one of the affected
          methods. This gives an overview over the commits that are not able to run with the problematic tools
          enabled.

    \item \textbf{Reflections:}\label{scanner:reflections}
          \\Although reflections are especially problematic, changes to reflectively accessed source code are
          hard to automatically detect. The class paths
          for reflective accesses are often only evaluated at runtime, e.g.~with the factory pattern. That is
          why we focused on detecting whether a project contains reflective accesses at all. This was done
          using a regular expression that detects \texttt{Class.forName} method calls. In a second step, we
          attempted to detect changes to classes that are actually accessed using reflections. As explained, this
          is not a trivial task. Our solution only supports hard coded class paths, leading to rather
          conservative estimations and incomplete evaluation results.
\end{itemize}
