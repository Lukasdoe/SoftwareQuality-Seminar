\section{Introduction}\label{sec:intro}

% What points should be in here?
% - Why should we care about Regression tests?
%   -> Regressions errors -> Open source -> Evolution
% - What is RTS and why is it required / important
%   -> less testing time -> more frequent testing possible and more immanent feedback for commiters
%   -> less ressources required for testing -> money
% - Approaches to reduce RT-Cost:
%     * Selection -> for multiple runs
%     * Reduction -> reducing test suits by eliminating redundant tests
%     * Prioritization -> Sorting test cases by probability of uncovering faults 

With agile and iterative software development processes, 
project owners need to ensure that every new revision of their program's source code still
complies with the project's specification. They need to make sure that no new bugs were
introduced. Ensuring the project's ongoing compliance is
increasingly complicated due to large, distributed code bases, the involvement of multiple developers and 
complex program
structures. Regression testing is widely used to detect such regression faults by
re-running existing tests to ensure that the 
new code changes have no negative impact on the code base.\cite{rt_in_agile,rt_iterative}

Executing all tests from a project's regression test suite is a time consuming task that also goes
with an economical cost. Even for companies with enormous computing resources, rerunning all
regression tests does not scale well. Google's \ac{tap} deals with an average program delivery frequency of one commit per
second, resulting in 150 million test runs a day~\cite{7965447}.

% This increase in maintenance is also the main 
% factors keeping
% project owners from utilizing automated testing in their projects.\cite{10.1155/2010/620836}
Regression Test Minimization, Regression Test Prioritization and \acf{rts} are the three major approaches
identified by Yoo and Harman (2012) to reduce the computational cost of regression testing~\cite{regression_testing_reduction}. 

% Short introduction to RTS, more than just the naming of the word
\ac{rts} aims to reduce the amount of
unnecessarily executed tests, i.e.~it excludes tests from the test suite whose outcome should not be
affected by the changes. This selection process is based on information about changes in the
program files and source code dependencies of tests cases. Only tests running changed code paths
are selected for execution. Other test cases should not change their behavior with the new code base.

\ac{rts}-tools exist in many variations. Though, one common goal of all examined tools is to be \emph{safe}.
% The test case dependency information can either be
% collected by \emph{statically} analyzing the program's source
% code or by executing the tests and \emph{dynamically} collecting coverage information. This
% information can be collected in different resolutions: Depending on the
% implementation, the test dependencies are collected per module or package, per file, class or method
% or in even smaller logical blocks, at a basic block or statement-level resolution~\cite{rts_techniques}.
An \ac{rts} system which does not exclude tests that can be affected by the applied changes is called
\emph{safe}~\cite{safe_definition}. Any undetected changes to dependencies of tests makes the \ac{rts}-tool \emph{unsafe},
as they open the possibility for regression faults to be overlooked by not running the corresponding
test cases~\cite{rts_techniques}.

Safety is hard to proof. Although many \ac{rts}-tools claim to be safe in most cases, this safety is
often only shown in experimental
evidence~\cite{unsafety_eval,gibstazi_paper} or in comparison to other \ac{rts}-tools~\cite{prestarts,starts_paper,gibstazi_paper}. Unsafety can be caused by methodological
limitations, errors in the implementation of the tool or outdated \ac{rts} software that is no
longer compatible with up-to-date library or language versions.\cite{unsafety_eval} Even if the \ac{rts}-tool's
methodology and implementation is proven to be safe in theory, unsafety can still occur through
third party language extensions such as the Spring Framework. 

With this seminar paper, we provide further insights into the unsafety of the following
open source \ac{rts}-tools for the Java programming language: The dynamic tools
\emph{Ekstazi}~\cite{ekstazimain,ekstazi_tool}, \emph{GIBstazi}~\cite{gibstazi_paper,gibstazi_tool},
\emph{OpenClover}~\cite{openclover_tool} and \emph{HyRTS}~\cite{hyrts_paper,hyrts_tool}, and
the static tool \emph{STARTS}~\cite{starts_paper,starts_tool}.

The last years show a shift in the open source software community away from the examined
\ac{rts}-tools, leading to an abandonment of their code bases. 
Even the often showcased~\cite{hyrts_paper,ekstazispec,ekstazimain} 
projects Apache CXF~\cite{cxf}, Commons Math~\cite{commonsmath}
and Camel~\cite{camel} stop integrating \emph{Ekstazi} into their testing process.\footnote{The
plugin was removed from Apache Camel with commit eb60743f22c2. Apache CXF and Commons Math do not
document a process that involves Ekstazi and did not update the plugin since its introduction in 2014.}

We show the potential and problems of the examined \ac{rts}-tools to help project owners
mitigate the risks and advantages of using \ac{rts} in their quality control procedures.
% In the first part of the study, we take a closer look at the
% examined \ac{rts}-tools in order to discover previously undocumented or unknown sources of unsafety.
% We searched for anomalies in the tools' source codes and problems originating from their general
% methodology. The resulting sources of unsafe behavior where then combined with already documented
% cases of unsafety and converted to model projects, showcasing them in a reproducible manner.
% For the second part of the research, we performed a large scale automated search on the source code of
% the most popular open source Java projects on GitHub. 
% With the data gathered from the laboratory
% evaluation of the \ac{rts}-tools, we declared a set of filters that were used for automatically
% identifying a subset of the discovered sources of unsafety.

We were able to identify 10 hypothetical scenarios that provoke unsafe behavior of at least one of the
examined \ac{rts}-tools. They include problems with reflections, dependency injection frameworks,
runtime instrumentation and external files. These scenarios were not just explained and executed in laboratory
conditions, but were also examined in the last 100 commits of a selection of 100 of the most popular open source GitHub
repositories. 

% - What do all RTS systems have in common? How do they work in general?
% - Short comparison of the evaluated systems and general approaches: 
%   -> static vs dynamic approaches
%   -> Granularity: File-Level vs. Class-Level vs. Module-Level vs. Method-Level vs. Basic-Block-Level
%   -> Online vs. Offline Modes -> Phases of RTS tools -> AE vs AEC
% - What is "safety" in terms of RTS tools? (most of tools refer to ekstazi as being the standard for "safe")
